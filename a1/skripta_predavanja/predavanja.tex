\documentclass[11pt]{article}

% ========== LANGUAGE AND ENCODING ==========
\usepackage[utf8]{inputenc}      % Allows UTF-8 input (e.g., ć, č, š)
\usepackage[T1]{fontenc}         % Correct output of special characters
\usepackage[serbian]{babel}      % Serbian language (Latin script by default)

% ========== MATH & UTILITIES ==========
\usepackage{amsmath,amssymb}
\usepackage{xcolor}      % only load xcolor once
\usepackage{xparse}
\usepackage{enumitem}
\usepackage{tcolorbox}
\tcbuselibrary{skins,breakable} % enable enhanced/breakable features
\usepackage{titlesec}

\titleformat{\section}[block]
  {\centering\normalfont\Huge\bfseries}
  {\thesection.}{1em}{}
\usepackage{makecell}
\usepackage{amsthm}
\usepackage{tikz}
\usetikzlibrary{math}
\usepackage{pgfplots}
\pgfplotsset{compat=1.18} % ensures modern pgfplots features


\pgfplotsset{compat=1.18}

\newcommand{\sekvenca}[6]{%
\begin{tikzpicture}[scale=1.1]
    % X-axis
    \pgfmathparse{#3 + 1} \let\xend\pgfmathresult
    \draw[->] (0,0) -- (\xend,0) node[right] {#4};
    % Y-axis
    \draw[->] (0,0) -- (0,5) node[above] {$a_{#4}$};

    % Y ticks
    \foreach \y in {0.5,1,1.5,2,2.5,3,3.5,4,4.5} {
        \draw (-0.1,\y) -- (0.1,\y) node[left=6pt] {\small \y};
    }

    % Sequence points
    \foreach \n in {#2,...,#3} {
        \pgfmathparse{#1}  % expression in \n
        \let\an\pgfmathresult
        \filldraw[blue] (\n,\an) circle (2pt);
    }

    % Lower horizontal line (if provided)
    \if\relax\detokenize{#5}\relax
        % nothing
    \else
      \draw[dashed, red] (0,#5) -- (\xend,#5) node[right] {\huge ${\text{ograničenje}} = #5$};
    \fi

    % Upper horizontal line (if provided)
    \if\relax\detokenize{#6}\relax
        % nothing
    \else
      \ifx#6+infty
        % do nothing if +infty
      \else
        \draw[dashed, red] (0,#6) -- (\xend,#6) node[right] {\huge ${\text{ograničenje}} = #6$};
      \fi
    \fi
\end{tikzpicture}
}



% ----- THEOREM STYLES -----

% Bold title, italic body (good for theorems)
% Theorem-like environments
\newcounter{mathitem} 
\renewcommand{\themathitem}{\arabic{mathitem}} % Format as 1, 2, 3, etc. (NO PERIOD HERE)

% Theorem-like environments all sharing the 'mathitem' counter
\theoremstyle{definition}
\newtheorem{definicija}[mathitem]{Definicija}
\newtheorem{primer}[mathitem]{Primer}   
\newtheorem{teorema}[mathitem]{Teorema}     
\newtheorem{tvrdjenje}[mathitem]{Tvrđenje}      
\makeatletter
\newtheoremstyle{serbnodot}% name
  {3pt}% space above
  {3pt}% space below
  {\normalfont}% body font
  {}% indent
  {\bfseries}% head font
  {}% punctuation after head (NO DOT)
  {0.5em}% space after head
  {\thmname{#1}}% head spec (only the name)
\makeatother

\theoremstyle{serbnodot}
\newtheorem*{dokaz}{Dokaz}

\DeclareMathOperator{\dx}{dx}
\renewcommand{\thesubsection}{\arabic{subsection}.}

% --- Section Format (Already centered and large) ---
\titleformat{\section}[block]
  {\centering\normalfont\Huge\bfseries}
  {} % Title is not numbered, so leave number field empty
  {0pt}
  {}

% --- Subsection Format (Centered and Bold) ---
\titleformat{\subsection}[block]
  {\centering\normalfont\Large\bfseries} % Centering, large font, bold
  {\thesubsection} % Use the new numbering (1., 2., 3.)
  {1em} % Space between number and title
  {}

% ========== CUSTOM COMMANDS ==========
\NewDocumentCommand{\naslov}{O{blue!70!black} m}{%
  \begin{center}
    \vspace{1em}
    {\color{#1}\Large\bfseries #2}
    \vspace{1em}
    \par
  \end{center}
}

\newcommand{\podnaslov}[1]{%
  \vspace{2em}
  {\color{black}\Large\bfseries #1}
  \vspace{0.5em}
  \par
}

% --- Zadatak counter ---
\newcounter{zadatak}[section]
\renewcommand{\thezadatak}{\arabic{zadatak}}

% --- Main zadatak environment ---
\NewDocumentEnvironment{zadatak}{O{} O{}}
{%
  \refstepcounter{zadatak}
  \begin{tcolorbox}[
    colback=blue!3,
    colframe=blue!40!black,
    sharp corners,
    boxrule=0.8pt,
    enhanced,
    breakable,
    title={\textbf{Zadatak \thezadatak\IfValueT{#1}{. #1}}},
    #2
  ]
}
{%
  \end{tcolorbox}
}


% --- Variant with parts (a), (b), ... ---
\NewDocumentEnvironment{zadatakp}{m O{}}
{%
  \refstepcounter{zadatak}
  \begin{tcolorbox}[
    colback=blue!3,
    colframe=blue!40!black,
    sharp corners,
    boxrule=0.8pt,
    enhanced,
    breakable,
    title={\textbf{Zadatak \thezadatak. #1}},
    #2
  ]
  \begin{enumerate}[label=\alph*)]
}
{%
  \end{enumerate}
  \end{tcolorbox}
}

% ========== PAGE STYLE ==========
\usepackage[a4paper, margin=2.5cm]{geometry}

% ========== DOCUMENT ==========
\begin{document}

\section*{Neodređeni integrali}

\subsection{Primitivna funkcija}

\begin{definicija}
Data je funkcija $ f:(a,b) \to \mathbb{R} $. Kažemo da je funkcija $ F:(a,b) \to \mathbb{R} $ \emph{primitivna funkcija} funkcije $ f $ na intervalu $ (a,b) $ ako je $ F $ diferencijabilna na $ (a,b) $ i ako važi $ F'(x) = f(x)$ $ \forall x \in (a,b) $.

\end{definicija}


\begin{primer}
Pronaći primitivnu funkciju funkcija:
\begin{enumerate}[label=\alph*)] % <-- This is the key change
    \item $ f:\mathbb{R} \to \mathbb{R}, \quad f(x) = 0, \quad  \forall x \in \mathbb{R} $ \\
    $F'(x) = f(x) = 0 \implies F(x) = c, c \in \mathbb{R}. $ 

    \item $ f:\mathbb{R} \to \mathbb{R}, \quad f(x) = 3x^2, \quad  \forall x \in \mathbb{R} $ \\
    $F'(x) = f(x) = 3x^2 \implies F(x) = x^3 + c, c \in \mathbb{R}. $ \\
    Moguće funkcije: $ F_1(x) = x^3, \quad F_2(x) = x^3 + \sqrt{3}, \quad F_3(x) = x^3 - 505 $ itd.

    \item $ f:\mathbb{R} \to \mathbb{R}, \quad f(x) = e^x, \quad  \forall x \in \mathbb{R} $ \\
    $F'(x) = f(x) = e^x \implies F(x) = e^x + c, c \in \mathbb{R}. $ 


    \item $ f:(0, +\infty) \to \mathbb{R}, \quad f(x) = \frac{1}{x}, \quad  \forall x \in (0,+\infty) $ \\
    $F'(x) = f(x) = \frac{1}{x} \implies F(x) = \ln|x| + c, c \in \mathbb{R}. $ 


  \end{enumerate}
\end{primer}


\begin{tvrdjenje}
    Ako je $ F:(a,b) \to \mathbb{R} $ primitivna funkcija funkcije $ f:(a,b) \to \mathbb{R} $, onda je i svaka funkcija oblika $ F(x) + c, \forall c \in \mathbb{R} $ primitivna funkcija funkcije $ f $.

    \begin{dokaz}
        \par\hangindent=3.5em
      Za neku primitivnu funkciju $ F $ funkcije $ f $ i za proizvoljan $ c \in \mathbb{R} $ posmatrajmo funkciju $ G:(a,b) \to \mathbb{R}, \quad G(x) = F(x) + c, \quad \forall x \in (a,b) $. Pošto je $ F $ diferencijabilna na $ (a,b) $, sledi da je i $ G $ diferencijabilna na $ (a,b) $ i važi:
      \[
        G'(x) = (F(x) + c)' = F'(x) + c' = f(x) + 0 = f(x), \quad \forall x \in (a,b).
      \]
      Dakle, $ G $ je primitivna funkcija funkcije $ f $.
    \end{dokaz}
  \end{tvrdjenje}

\begin{tvrdjenje}
  Neka su $F_1, F_2:(a,b) \to \mathbb{R} $ primitivne funkcije funkcije $ f:(a,b) \to \mathbb{R} $. Tada postoji $ c \in \mathbb{R} $ takav da je $ F_1(x) = F_2(x) + c, \quad \forall x \in \mathbb{R} $.

  \begin{dokaz}
    \par\hangindent=3.5em
    Za funkciju $ f $ imamo dve njene primitivne funkcije $ F_1^{'}(x) = f(x) $ i $f$: $ F_2^{'}(x) = f(x) $. Posmatrajmo funkciju $ G:(a,b) \to \mathbb{R}, \quad G(x) = F_1(x) - F_2(x), \quad \forall x \in (a,b) $. Pošto su $ F_1 $ i $ F_2 $ diferencijabilne na $ (a,b) $, sledi da je i $ G $ diferencijabilna na $ (a,b) $ i važi:
    \[
      G'(x) = (F_1(x) - F_2(x))' = F_1'(x) - F_2'(x) = f(x) - f(x) = 0, \quad \forall x \in (a,b).
    \]
    Iz ovoga sledi da je funkcija $ G $ konstantna na intervalu $ (a,b) $, tj. postoji $ c \in \mathbb{R} $ takav da je $ G(x) = c, \quad \forall x \in (a,b) $.
  \end{dokaz}
\end{tvrdjenje}

\begin{definicija}
  Skup svih primitivnih funkcija funkcije $ f:(a,b) \to \mathbb{R} $ označavamo sa $ \int f(x) \, dx $ i nazivamo ga \emph{neodređeni integral} funkcije $ f $. Dakle,
  \[
    \int f(x) \, dx = \{ F(x) + c \mid c \in \mathbb{R} \},
  \]
  gde je $ F:(a,b) \to \mathbb{R} $ neka primitivna funkcija funkcije $ f $.
\end{definicija}

\begin{tabular}{ |c|c| }
    \hline
    \multicolumn{2}{|c|}{Tablica integrala:}                                                                                                                                                                                                        \\ \hline
    $\displaystyle\alpha \in \mathbb{R} \backslash \left\{-1\right\},\ x\in\left(0, +\infty\right)\quad \int x^{\alpha} \dx$                   & $\displaystyle\frac{1}{\alpha + 1} x^{\alpha + 1} + C$                                             \\ \hline
    $\displaystyle n \in \mathbb{N},\ x \in \mathbb{R} \quad \int x^{n} \dx$                                                                   & $\displaystyle\frac{1}{n + 1} x^{n + 1} + C$                                                       \\ \hline
    $\displaystyle n \in \mathbb{N}\backslash\left\{1\right\},\ x \in\mathbb{R}\backslash\left\{0\right\}\quad \int x^{-n} \dx $               &
    \makecell{
    $\displaystyle\frac{1}{1-n} x^{1-n} + C_1,\ x \in \left(-\infty, 0\right)$                                                                                                                                                                      \\
        $\displaystyle\frac{1}{1-n} x^{1-n} + C_2,\ x \in \left(0, +\infty\right)$
    }                                                                                                                                                                                                                                               \\  \hline
    $\displaystyle x \in \mathbb{R} \quad \int \frac{1}{x} \dx$                                                                                     &
    \makecell{ $\ln |x| + C_1,\ x \in \left(-\infty, 0\right)$                                                                                                                                                                                      \\
    $\displaystyle\ln |x| + C_2,\ x \in \left(0, +\infty\right)$}                                                                                                                                                                                   \\ \hline
    $\displaystyle a > 0,\ a\neq 1,\ x \in \mathbb{R} \quad \int a^x \dx$                                                                      & $ \displaystyle\frac{1}{\ln a} a^x + C$                                                            \\ \hline
    $\displaystyle x \in \mathbb{R} \quad \int \sin x \dx$                                                                                     & $ \displaystyle-\cos x + C$                                                                        \\ \hline
    $\displaystyle x \in \mathbb{R} \quad \int \cos x \dx$                                                                                     & $ \displaystyle\sin x + C$                                                                         \\ \hline
    $\displaystyle x \in \bigcup_{k \in \mathbb{Z}} \left(\frac{\pi}{2} + k\pi, \frac{3\pi}{2} + k\pi\right)\quad \int \frac{1}{\cos^2 x} \dx$ & $ \tg x + C_k,\ x \in \left(\frac{\pi}{2} + k\pi, \frac{3\pi}{2} + k\pi\right),\ k \in \mathbb{Z}$ \\ \hline
    $\displaystyle x \in \bigcup_{k \in \mathbb{Z}} \left(k\pi, \pi + k\pi\right)\quad \int \frac{1}{\sin^2 x} \dx$                            & $ \displaystyle-\ctg x + C_k,\ x \in \left(k\pi, \pi + k\pi\right),\ k \in \mathbb{Z}$             \\ \hline
    $\displaystyle x \in \left(-1, 1\right) \quad \int \frac{1}{\sqrt{\left(1 - x^2\right)}} \dx                      $                        & $ \arcsin x + C$                                                                                   \\ \hline
    $\displaystyle x \in \mathbb{R} \quad \int \frac{1}{1+x^2} \dx$                                                                            & $ \arctg x + C$                                                                                    \\ \hline
    $\displaystyle x \in \mathbb{R}\backslash \left\{0\right\} \quad \int x^{0} \dx$                                                           &
    \makecell{
    $x + C_1,\ x \in \left(-\infty, 0\right)$                                                                                                                                                                                                       \\
        $x + C_2,\ x \in \left(0, +\infty\right)$
    }                                                                                                                                                                                                                                               \\ \hline
\end{tabular}

\begin{tvrdjenje}{\textbf{Linearnost neodređenog integrala.}}
  Neka su $ f,g:(a,b) \to \mathbb{R} $ funkcije koje imaju primitivne funkcije na intervalu $ (a,b) $. Tada i $ \lambda f(x) + \mu g(x) $ ima primitivnu funkciju na intervalu $ (a,b) $ za sve $ \lambda, \mu \in \mathbb{R} $ i važi:
  \[
    \int (\lambda f(x) + \mu g(x)) \, dx = \lambda \int f(x) \, dx + \mu \int g(x) \, dx.
  \]
  
  \begin{dokaz}
    
    \par\hangindent=3.5em
    Neka su $ F $ i $ G $ primitivne funkcije funkcija $ f $ i $ g $, redom. Hoćemo da pokažemo da je funckija $ \lambda F + \mu G $ primitivna za $ \lambda f + \mu g $. 
    \begin{align*}
      (\lambda F(x) + \mu G(x))' &= \lambda F'(x) + \mu G'(x) = \lambda f(x) + \mu g(x). \\
      & = \int (\lambda f(x) + \mu g(x)) \, dx = \lambda F(x) + \mu G(x) + C, \quad C \in \mathbb{R}.\\
      & = \lambda \int f(x) \, dx + \mu \int g(x) \, dx.
\end{align*}
  \end{dokaz}
\end{tvrdjenje}

\begin{tvrdjenje} {\textbf{Parcijalna integracija.}} Neka su $u, v:(a,b) \to \mathbb{R} $ diferencijabilne funkcije na intervalu $ (a,b) $. Tada važi:
\[
  \int u(x)v'(x) \, dx = u(x)v(x) - \int v(x)u'(x) \, dx.
\]
  
  \begin{dokaz}
    \par\hangindent=3.5em
    \begin{align*}
      (u(x)v(x))' &= u'(x)v(x) + u(x)v'(x) \\
      \intertext{Zatim integral sa obe strane:} \\
      \int (u(x)v(x))' \, dx &= \int u'(x)v(x) \, dx + \int u(x)v'(x) \, dx \\
      \intertext{Izdvajamo:$ \int u(x)v'(x)dx:$} 
      \int u(x)v'(x) \, dx &= \int (u(x)v(x))' \, dx - \int u'(x)v(x) \, dx \\
      &= u(x)v(x) - \int v(x)u'(x) \, dx.
    \end{align*}
  \end{dokaz}


\end{tvrdjenje}

\begin{primer}
  \begin{align*}
    \int \log x \, dx 
    &= 
    \left\{
    \begin{array}{l}
    u(x) = \log x \;\Rightarrow\; du = \frac{1}{x} dx \\
    dv = dx \;\Rightarrow\; v(x) = x
    \end{array}
    \right\} \\
    &= u v - \int v \, du \\
    &= x \log x - \int x \cdot \frac{1}{x} \, dx \\
    &= x \log x - \int dx \\
    &= x \log x - x + c, \quad c \in \mathbb{R}.
\end{align*}

\end{primer}


\begin{primer}
  \begin{align*}
    \int \underbrace{e^x}_{u} \, \underbrace{\cos x \, dx}_{dv}
    &= 
    \left\{
      \begin{array}{l}
          u = e^x \;\Rightarrow\; du = e^x \, dx \\
          dv = \cos x \, dx \;\Rightarrow\; v = \sin x
      \end{array}
    \right\} \\
    &= e^x \sin x - \int e^x \sin x \, dx \\
    &= e^x \sin x - 
    \int \underbrace{e^x}_{u} \, \underbrace{\sin x \, dx}_{dv}
    =
    \left\{
      \begin{array}{l}
          u = e^x \;\Rightarrow\; du = e^x \, dx \\
          dv = \sin x \, dx \;\Rightarrow\; v = -\cos x
      \end{array}
    \right\} \\
    &= e^x \sin x - \left(-e^x \cos x + \int e^x \cos x \, dx \right) \\
    \intertext{Ovde uvodimo oznaku $ I = \int e^x \cos x \, dx $ pošto se ponavlja isti integral kao početni:}
    I &= e^x \sin x + e^x \cos x - I \\
    2I &= e^x (\sin x + \cos x) \\
    I &= \frac{e^x (\sin x + \cos x)}{2} + c,\qquad c \in \mathbb{R}.
\end{align*}



\end{primer}

\begin{tvrdjenje}
  Neka je data funkcija $ f:(a,b) \to \mathbb{R} $  i njena primitivna funkcija $ F:(a,b) \to \mathbb{R} $. Ako je funkcija $ \rho:(\alpha, \beta) \to (a,b) $ diferencijabilna na intervalu $ (\alpha, \beta) $, tada važi:
  \[
    \int f(\rho(x)) \cdot \rho'(x) \, dt = F(\rho(x)) + c, \quad c \in \mathbb{R}. \tag{1}
  \]
  \begin{dokaz}
    \par\hangindent=3.5em
    Proverimo da li je izvod funkcije $ F(\rho(x)) + C $ jednak funkciji $ f(\rho(t)) \cdot \rho'(t) $ (tj. funckiji pod integralom):
  
    \[ \left( F(\rho(x)) + C \right)' = (F\circ\rho)'(x) = F'(\rho(x)) \cdot \rho'(x) = f(\rho(x)) \cdot \rho'(x) = (1)\]
  
  
  \end{dokaz}

\end{tvrdjenje}

\begin{primer}
  \begin{align*}
    \int \frac{\dx}{x+3} &= \left\{ \begin{array}{l}
      t = x + 3, \quad dt = dx, \quad x > -3 \\
      \text{ili } x < -3, \quad dt = dx
      \end{array}
      \right\} \\
    &= \int \frac{dt}{t} = \ln|t| + c = \ln|x+3| + c, \quad c \in \mathbb{R}.
  \end{align*}
\end{primer}

\begin{primer}
  \begin{align*}
    \int \frac{x^2}{1+x^3} \dx &= \left\{ \begin{array}{l}
      t = 1 + x^3, \\ dt = 3x^2 \, dx, \\
      dt \cdot \frac{1}{3} = x^2 \, dx.
      \end{array}
      \right\} \\
    &= \int \frac{1}{t} \cdot \frac{1}{3} \cdot dt = \frac{1}{3} \int \frac{dt}{t} = \frac{1}{3} \ln|t| + c = \frac{1}{3} \ln|1+x^3| + c, \quad c \in \mathbb{R}.
  \end{align*}
\end{primer}

\subsection{Tehnike integracija}

\end{document}


